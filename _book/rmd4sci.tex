\documentclass[]{article}
\usepackage{lmodern}
\usepackage{amssymb,amsmath}
\usepackage{ifxetex,ifluatex}
\usepackage{fixltx2e} % provides \textsubscript
\ifnum 0\ifxetex 1\fi\ifluatex 1\fi=0 % if pdftex
  \usepackage[T1]{fontenc}
  \usepackage[utf8]{inputenc}
\else % if luatex or xelatex
  \ifxetex
    \usepackage{mathspec}
  \else
    \usepackage{fontspec}
  \fi
  \defaultfontfeatures{Ligatures=TeX,Scale=MatchLowercase}
\fi
% use upquote if available, for straight quotes in verbatim environments
\IfFileExists{upquote.sty}{\usepackage{upquote}}{}
% use microtype if available
\IfFileExists{microtype.sty}{%
\usepackage{microtype}
\UseMicrotypeSet[protrusion]{basicmath} % disable protrusion for tt fonts
}{}
\usepackage[margin=1in]{geometry}
\usepackage{hyperref}
\hypersetup{unicode=true,
            pdfborder={0 0 0},
            breaklinks=true}
\urlstyle{same}  % don't use monospace font for urls
\usepackage{natbib}
\bibliographystyle{plainnat}
\usepackage{longtable,booktabs}
\usepackage{graphicx,grffile}
\makeatletter
\def\maxwidth{\ifdim\Gin@nat@width>\linewidth\linewidth\else\Gin@nat@width\fi}
\def\maxheight{\ifdim\Gin@nat@height>\textheight\textheight\else\Gin@nat@height\fi}
\makeatother
% Scale images if necessary, so that they will not overflow the page
% margins by default, and it is still possible to overwrite the defaults
% using explicit options in \includegraphics[width, height, ...]{}
\setkeys{Gin}{width=\maxwidth,height=\maxheight,keepaspectratio}
\IfFileExists{parskip.sty}{%
\usepackage{parskip}
}{% else
\setlength{\parindent}{0pt}
\setlength{\parskip}{6pt plus 2pt minus 1pt}
}
\setlength{\emergencystretch}{3em}  % prevent overfull lines
\providecommand{\tightlist}{%
  \setlength{\itemsep}{0pt}\setlength{\parskip}{0pt}}
\setcounter{secnumdepth}{5}
% Redefines (sub)paragraphs to behave more like sections
\ifx\paragraph\undefined\else
\let\oldparagraph\paragraph
\renewcommand{\paragraph}[1]{\oldparagraph{#1}\mbox{}}
\fi
\ifx\subparagraph\undefined\else
\let\oldsubparagraph\subparagraph
\renewcommand{\subparagraph}[1]{\oldsubparagraph{#1}\mbox{}}
\fi

%%% Use protect on footnotes to avoid problems with footnotes in titles
\let\rmarkdownfootnote\footnote%
\def\footnote{\protect\rmarkdownfootnote}

%%% Change title format to be more compact
\usepackage{titling}

% Create subtitle command for use in maketitle
\newcommand{\subtitle}[1]{
  \posttitle{
    \begin{center}\large#1\end{center}
    }
}

\setlength{\droptitle}{-2em}

  \title{}
    \pretitle{\vspace{\droptitle}}
  \posttitle{}
    \author{}
    \preauthor{}\postauthor{}
    \date{}
    \predate{}\postdate{}
  
\usepackage{booktabs}

\begin{document}

{
\setcounter{tocdepth}{2}
\tableofcontents
}
\hypertarget{about-this}{%
\section{About this}\label{about-this}}

This is a 3 hour workshop initially designed for the SSA Vic and MIG
meeting on November 13, 2018.

After completing this course, you will know how to:

\begin{itemize}
\tightlist
\item
  Create your own R Markdown document
\item
  Create figures and tables that you can reference in text, and update
  with - your data
\item
  Export your R Markdown document to PDF, HTML, and Microsoft Word
\item
  Use keyboard shortcuts to improve workflow
\item
  Cite research articles and generate a bibliography
\end{itemize}

We may, depending on time, also cover the following areas:

\begin{itemize}
\tightlist
\item
  Change the size and type of your figures
\item
  Create captions for your figures, and reference them in text
\item
  Cite research articles and generate a bibliography
\item
  Debug and handle common errors
\end{itemize}

\hypertarget{getting-oriented}{%
\section{Getting oriented}\label{getting-oriented}}

\hypertarget{overview}{%
\subsection{Overview}\label{overview}}

\hypertarget{questions}{%
\subsubsection{Questions}\label{questions}}

\hypertarget{software-setup}{%
\subsection{Software Setup}\label{software-setup}}

\hypertarget{r}{%
\subsection{R}\label{r}}

\hypertarget{windows}{%
\subsubsection{Windows}\label{windows}}

\hypertarget{macos}{%
\subsubsection{MacOS}\label{macos}}

\hypertarget{rstudio}{%
\subsection{RStudio}\label{rstudio}}

\hypertarget{windows-1}{%
\subsubsection{Windows}\label{windows-1}}

\hypertarget{macos-1}{%
\subsubsection{MacOS}\label{macos-1}}

\hypertarget{rmarkdown}{%
\subsection{RMarkdown}\label{rmarkdown}}

\hypertarget{windows-2}{%
\subsubsection{Windows}\label{windows-2}}

\hypertarget{macos-2}{%
\subsubsection{MacOS}\label{macos-2}}

\hypertarget{tex-and-tinytex}{%
\subsection{TeX (and TinyTeX)}\label{tex-and-tinytex}}

\hypertarget{windows-3}{%
\subsubsection{Windows}\label{windows-3}}

\hypertarget{macos-3}{%
\subsubsection{MacOS}\label{macos-3}}

\hypertarget{why-r-rmarkdown}{%
\section{Why R / RMarkdown}\label{why-r-rmarkdown}}

\hypertarget{overview-1}{%
\subsection{Overview}\label{overview-1}}

\hypertarget{questions-1}{%
\subsection{Questions}\label{questions-1}}

\hypertarget{objectives}{%
\subsection{Objectives}\label{objectives}}

\hypertarget{why-are-we-here}{%
\subsection{Why are we here?}\label{why-are-we-here}}

\begin{enumerate}
\def\labelenumi{\arabic{enumi}.}
\tightlist
\item
  Form small groups of 2-4 with your neighbours and discuss how you
  expect learning rmarkdown might benefit you.
\end{enumerate}

\hypertarget{getting-started}{%
\section{Getting Started}\label{getting-started}}

\hypertarget{overview-2}{%
\subsection{Overview}\label{overview-2}}

\hypertarget{questions-2}{%
\subsection{Questions}\label{questions-2}}

\hypertarget{objectives-1}{%
\subsection{Objectives}\label{objectives-1}}

\hypertarget{exercise}{%
\subsection{Exercise}\label{exercise}}

\begin{enumerate}
\def\labelenumi{\arabic{enumi}.}
\tightlist
\item
  Generic exercise
\end{enumerate}

\hypertarget{figures-and-tables}{%
\section{Figures and Tables}\label{figures-and-tables}}

\hypertarget{overview-3}{%
\subsection{Overview}\label{overview-3}}

\hypertarget{questions-3}{%
\subsection{Questions}\label{questions-3}}

\hypertarget{objectives-2}{%
\subsection{Objectives}\label{objectives-2}}

\hypertarget{exercise-1}{%
\subsection{Exercise}\label{exercise-1}}

\begin{enumerate}
\def\labelenumi{\arabic{enumi}.}
\tightlist
\item
  Generic exercise
\end{enumerate}

\hypertarget{html-pdf-and-word}{%
\section{HTML, PDF, and Word}\label{html-pdf-and-word}}

\hypertarget{overview-4}{%
\subsection{Overview}\label{overview-4}}

\hypertarget{questions-4}{%
\subsection{Questions}\label{questions-4}}

\hypertarget{objectives-3}{%
\subsection{Objectives}\label{objectives-3}}

\hypertarget{exercise-2}{%
\subsection{Exercise}\label{exercise-2}}

\begin{enumerate}
\def\labelenumi{\arabic{enumi}.}
\tightlist
\item
  Generic exercise
\end{enumerate}

\hypertarget{citations}{%
\section{Citations}\label{citations}}

\hypertarget{overview-5}{%
\subsection{Overview}\label{overview-5}}

\hypertarget{questions-5}{%
\subsection{Questions}\label{questions-5}}

\hypertarget{objectives-4}{%
\subsection{Objectives}\label{objectives-4}}

\hypertarget{exercise-3}{%
\subsection{Exercise}\label{exercise-3}}

\begin{enumerate}
\def\labelenumi{\arabic{enumi}.}
\tightlist
\item
  Generic exercise
\end{enumerate}

\hypertarget{changing-the-size-and-type-of-your-figures}{%
\section{Changing the size and type of your
figures}\label{changing-the-size-and-type-of-your-figures}}

\hypertarget{overview-6}{%
\subsection{Overview}\label{overview-6}}

\hypertarget{questions-6}{%
\subsection{Questions}\label{questions-6}}

\hypertarget{objectives-5}{%
\subsection{Objectives}\label{objectives-5}}

\hypertarget{exercise-4}{%
\subsection{Exercise}\label{exercise-4}}

\begin{enumerate}
\def\labelenumi{\arabic{enumi}.}
\tightlist
\item
  Generic exercise
\end{enumerate}

\hypertarget{creating-captions}{%
\section{Creating Captions}\label{creating-captions}}

\hypertarget{common-problems}{%
\section{Common Problems}\label{common-problems}}

\hypertarget{overview-7}{%
\subsection{Overview}\label{overview-7}}

\hypertarget{questions-7}{%
\subsection{Questions}\label{questions-7}}

\hypertarget{objectives-6}{%
\subsection{Objectives}\label{objectives-6}}

\hypertarget{exercise-5}{%
\subsection{Exercise}\label{exercise-5}}

\begin{enumerate}
\def\labelenumi{\arabic{enumi}.}
\tightlist
\item
  Generic exercise
\end{enumerate}

\hypertarget{learning-more}{%
\section{Learning More}\label{learning-more}}


\end{document}
